\documentclass[12pt]{article}
\usepackage[utf8]{inputenc}

\usepackage[margin=1in]{geometry}
\usepackage{lipsum}

\usepackage[backend=biber,style=ieee]{biblatex}
\addbibresource{sources.bib}

\usepackage{titling}
\newcommand{\subtitle}[1]{%
	\posttitle{%
		\par\end{center}
	\begin{center}\large#1\end{center}
	\vskip0.1em}}%

\title{HW 4 Reading Analysis\\
NRDC v. FAA}
\subtitle{PEGN 430A}
\author{Tyler Singleton}
\date{01 March 2022}

\begin{document}
\maketitle

\newpage
\setlength{\parindent}{0pt}

% --- Questions Section --- %
\textbf{Questions} \\

% Question 1
\textbf{1. Who are the parties?} \\

The major parties within this case include: The Natural Resources Defense Council (NRDC), Defenders of Wildlife, and Friends of PFN acting as petitioners; and the Federal Aviation Administration (FAA) acting as respondents (pg 1). 
\\

% Question 2
\textbf{2. What happened procedurally that now allows them to be in this court?} \\

From \textsection 1(B)(1-3), Panama City Airport needed to expand to due to updated FAA regulations. Panama City Airport considered a local expansion project, but found this to not be suitable, so they prepared to utilize land donated to them. This site is referenced as the West Bay Site. Panama City Airport asked the FAA to approve development of the new location. FAA required Panama City Airport to preform an Environmental Impact Statement (EIS) in accordance to National Environmental Policy Act (NEPA). After public comment, FAA approved the development. \\

Continuing from \textsection 1(B)(1-3), the petitioners asked the court to stay the FAA's approval with the reason that the FAA does not have authority to approve projects with significant environmental impact because of the AAIA. \\

% Question 3
\textbf{3. What part of NEPA is being challenged in this case?  Be specific.} \\

\textsection (1)(C)(2)(a) lays out the specific part of NEPA being challenged falls under 40 C.F.R. \textsection 1502.14 where it describes that the agency must consider all reasonable alternatives objectively. The case documents that petitioners argued the FAA failed to adequately follow this section in reviewing alternatives by overlooking another proposed site such as Goose Bayou, did not go into detail. They further elaborate to say the FAA acted arbitrarily when screening out the reasonable alternatives for review. \\

% Question 4
\textbf{4. What is the “standard of review” that the Court used in this case?} \\

Some text \\

% Question 5
\textbf{5. What are the conclusions of the case? In other words, what is the holding of the Court?} \\

More text \\

\end{document}
